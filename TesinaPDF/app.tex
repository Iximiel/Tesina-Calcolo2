\section{Applicazione}
\subsection{Equazione del calore}
Riprendiamo l'equazione del calore:
\begin{equation}
\pde T t(x,t) =k\pde{^2}{x^2}T(x,t)
\end{equation}

Per rispettare la convenzione di prima $D_2 = k$ , $D_1 = U = V(x,t) = 0$. 


A questo punto posso sostituire, con $\eta = k\frac{\Delta t}{\Delta x^2}$:
\begin{equation}\label{eq:pararametriHeat}
\begin{array}{ll}
a_i^j = -1            & ak_i^j =1\\
d_i^j = \frac2\eta +2 & dk_i^j = \frac2\eta -2 \\
c_i^j = -1             & ck_i^j =1\\
e_i^j = 0             & ek_i^j =0
\end{array}
\end{equation}
e procedere con i calcoli.
\subsection{Equazione di \Schrodinger}
Lavoriamo con l'equazione di \Schrodinger  dipendente dal tempo:
\begin{equation}
i\hbar\pde \psi t(x,t) =\lrq{-\frac{\hbar^2}{2m}\pde{^2}{x^2}+V(x,t)} \psi(x,t)
\end{equation}
prima di tutto portiamola in una forma tale che non ci sia nulla a moltiplicare la derivata temporale:
\begin{equation}
\pde \psi t(x,t) =\lrq{i\frac{\hbar}{2m}\pde{^2}{x^2}+\frac{\varLambda(x,t)}{i\hbar}} {\psi(x,t)}
\end{equation} 

Per rispettare la convenzione di prima $D_2 = i\frac{\hbar}{2m}$ , $D_1 = U = 0$ e $V(x,t) = \frac{\varLambda(x,t)}{i\hbar}$. Ho usato $\varLambda$ per indicare il potenziale in modo da evitare confusione.


A questo punto posso sostituire, con $\eta = i\frac{\hbar}{2m}\frac{\Delta t}{\Delta x^2}$:
\begin{equation}\label{eq:pararametriSC}
\begin{array}{ll}
a_i^j = -1            & ak_i^j =1\\
d_i^j = \frac1\eta\lrt{2-\Delta t V_i^{j+1}} +2 & dk_i^j = \frac1\eta\lrt{2+\Delta tV_i^{j}} -2 \\
c_i^j = -1             & ck_i^j =1\\
e_i^j = 0             & ek_i^j =0
\end{array}
\end{equation}