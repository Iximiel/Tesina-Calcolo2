\section{Introduzione: matematica}
\subsection{Derivate numeriche}
Prima di tutto faccio un piccolo elenco di metodi di calcolo per le derivate, usando il metodo delle differenze finite:

La derivata prima (in avanti) \`e:
\begin{equation}
  \pde{F}x(a) \simeq \frac{F(a+h)-F(a)}{h} + O(h)
\end{equation}
Ma \`e al primo ordine, per una maggiore precisione si pu\`o utilizzare la cosiddetta definizione ``centrale'':
\begin{equation}
  \pde{F}x(a) \simeq \frac{F(a+h)-F(a-h)}{2*h} + O(h^2)
\end{equation}

Per la derivata seconda il discorso \`e simile ma in entrambi i casi la precisione \`e sempre al secondo ordine:
\begin{equation}
  \pde{^2F}{x^2}(a) \simeq \frac{\pde{F}x(a+h)-\pde{F}x(a)}{h} = \frac{F(a+2h)+F(a)-2F(a+h)}{h^2} + O(h^2)
\end{equation}

\begin{equation}
  \pde{^2F}{x^2}(a) \simeq  \frac{F(a+h)+F(a-h)-2F(a)}{h^2} + O(h^2)
\end{equation}

Definito come si calcolano le derivate che ci interessano posso mostrare come utilizzarle per risolvere numericamente equazioni differenziali.

\subsection{Equazione del calore}
Il metodo che voglio illustrare inizialmente \`e stato concepito per la risoluzione di equazioni differenziali iperboliche, come quella del calore:
\begin{equation}\label{eq:calore}
  \pde Tt =K \pde{^2T}{x^2}
\end{equation}

La prima cosa da fare \`e calcolare i valori delle derivate:

\begin{equation}
  \pde Tt\lrt{x,t} = \frac{T(x,t+\Delta t)-T(x,t)}{\Delta t}
\end{equation}
e
\begin{equation}
  \pde{^2T}{x^2}(x,t) = \frac{T(x+\Delta x,t)+T(x-\Delta x,t)-2T(x,t)}{\Delta x^2}
\end{equation}
\subsubsection{Eulero esplicito, ``in avanti''}
Quindi prendo i risultati e li sostituisco nell'equazione le derivate (con queste definizioni si ha precisione $\Delta x^2$ nello spazio e $\Delta t$ nel tempo):
\begin{equation}
  \frac{T(x,t+\Delta t)-T(x,t)}{\Delta t} = k \frac{T(x+\Delta x,t)+T(x-\Delta x,t)-2T(x,t)}{\Delta x^2}
\end{equation}
e andando avanti:
\begin{equation}
  T(x,t+\Delta t) = k\frac{\Delta t}{\Delta x^2} \lrt{T(x+\Delta x,t)+T(x-\Delta x,t)-2T(x,t)} + T(x,t)
\end{equation}

A questo punto calcolo ogni punto conoscendo il corrispondente e i due primi vicini del passo precedente.
\subsubsection{Eulero implicito, ``all'indietro''}
Posso, volendo, calcolare la derivata spaziale nell'istante temporale successivo:
\begin{equation}
  \frac{T(x,t+\Delta t)-T(x,t)}{\Delta t} = k \frac{T(x+\Delta x,t+\Delta t)+T(x-\Delta x,t+\Delta t)-2T(x,t+\Delta t)}{\Delta x^2}
\end{equation}
proseguo:
\begin{equation}
  T(x,t+\Delta t)- k \frac{\Delta t}{\Delta x^2}\lrt{T(x+\Delta x,t+\Delta t)+T(x-\Delta x,t+\Delta t)-2T(x,t+\Delta t)} = T(x,t)
\end{equation}

Anche in questo caso non approfondisco, l'approccio per la soluzione \`e simile al metodo di Crank-Nicolson, ma preferisco parlarne direttamente di quest'ultimo per poi generalizzarlo e descriverne l'algoritmo utilizzato per la risoluzione dell'equazione di \Schrodinger

\subsection{Un esempio di come ricavare il metodo di Crank Nicolson: l'equazione del calore}
Innanzitutto utilizziamo la definizione centrale della derivata prima in modo da avere una precisione di $\Delta t^2$ anche per quanto riguarda il tempo, ma  calcolandola in $t+\Delta t/2$ con incremento $\Delta t/2$:
\begin{equation}
  \pde{T}t(x,t+\Delta t/2) \simeq \frac{T(x,t+\Delta t/2+\Delta t/2)-T(x,t-\Delta t/2+\Delta t/2)}{2 \Delta t/2} = \frac{T(x,t+\Delta t)-T(x,t)}{\Delta t}
\end{equation}
Per lo spazio utilizziamo le derivate calcolate negli esempi precedenti.
A questo punto abbiamo $\pde{T}t(x,t+\Delta t/2)$, $\pde{^2T}{x^2}(x,t)$ e $\pde{^2T}{x^2}(x,t+\Delta t)$ per calcolare l'equazione facciamo la media delle due derivate spaziali ai tempi $t$ e $t+\Delta t$.
\begin{equation}\label{eq:HeatForCrank}
  \begin{aligned}
    \frac{T(x,t+\Delta t)-T(x,t)}{\Delta t} = \frac K2 &\lr(.{\frac{T(x+\Delta x,t)+T(x-\Delta x,t)-2T(x,t)}{\Delta x^2}}\\
    &\lr.){+\frac{T(x+\Delta x,t+\Delta t)+T(x-\Delta x,t+\Delta t)-2T(x,t+\Delta t)}{\Delta x^2}}
  \end{aligned}
\end{equation}

In pochi passaggi si arriva a separare le parti a tempo differente:
\begin{equation}
  \lrt{\frac 2\eta +2}T(x,t+\Delta t) -T(x+\Delta x,t+\Delta t)-T(x-\Delta x,t+\Delta t) = \lrt{\frac 2\eta -2}T(x,t)+T(x+\Delta x,t)+T(x-\Delta x,t)
\end{equation}
 dove $\eta = K\frac{\Delta t}{\Delta x^2}$.

Per ogni istante di tempo ho una matrice tridiagonale per il passo temporale che conosco e una per quello successivo. In \autoref{section:soluzionetri} descriver\`o come si risolve una di queste matrici.
