\section{Varie}

\pgfplotscreateplotcyclelist{mylist}{%
	{green},{red},{orange},{blue}
} 

\begin{figure}
	\centering
	\begin{tikzpicture}
	\def \mya {2};
	\def \mys {2};
	\def \myx {2};
	\def \mycol {black};
	\begin{axis}[axis x line = center, axis y line = center, xmin = -0.1, xmax = 10, ymin=-0.1,ymax =1.1, cycle list name=mylist]
		\addplot[domain=0:10, samples =501, black]	plot[id=barrdemo]	function{(x>\mya?1:0)*(x<(\mya+\mys)?1:0)};
		\foreach \mys in {0.5,1,2,4}{
			\expandafter\addplot[domain=0:10, samples =501]	plot[id=gauss]	function{exp(-(x-\myx)*(x-\myx)/(2*\mys^2))};
		}
	\end{axis}
\end{tikzpicture}
	\caption{In questa figura mostro il confronto tra una barriera del tipo \eqref{eq:Barriera} con $a=2$ e una diverse gaussiane}
\end{figure}

%%%%%%%%%%%%%%%%%%%%%%%%%%%%%%%%%%%%%%%%%%%%%%%%%%%%%%%%%%%%%%%%%%%%%%%%%%%%%%%%%%%%%%%%%%%%%%%

\subsection{Calcoli Salto}\label{sec:Salto}
Procedo con il risolvere l'equazione di \Schrodinger con un potenziale del tipo \eqref{eq:Barriera}.
Prendiamo un dominio infinito con in 0 il salto di potenziale, l'equazione, indipendente dal tempo, avr\`a la forma:
\begin{equation}
E\psi = \lrt{-\frac{\hbar^2}{2m}\pde{^2}{x^2}+V_0H(x)}\psi
\end{equation}
Divido lo spazio in due porzioni ($x<0$ e $x>0$) in cui:
\begin{equation}\label{eq:Ks}
\begin{array}{lr}
k_1 = \sqrt\frac{2m E}{\hbar^2}&x<0\\
k_2 = \sqrt\frac{2m \lrt{E-V_0}}{\hbar^2}&x>0
\end{array}
\end{equation}
Se per esempio avessimo delle onde piane in ciascuna porzione dello spazio ne avrei una che si muove verso destra ($\psi_\to$) e una verso sinistra ($\psi_\from$) e quindi in 0 avrei, chiamando i coefficienti delle onde piane $A$ per il lato sinistro e $B$ quello destro:
\begin{equation}
\psi\lrt x\lr\{.{\begin{array}{lr}
	A_\to e^{ik_1x} + A_\from e^{-ik_1x}	&	x<0\\
	B_\to e^{ik_2x} + B_\from e^{-ik_2x}	&	x>0
	\end{array}}
\end{equation}

Dato che il potenziale ha una discontinuit\`a finita in $0$, funzione e derivata devono essere continue:
\begin{equation}
\lr\{.{\begin{array}{l}
	A_\to+A_\from = B_\to+B_\from\\
	k_1\lrt{A_\to-A_\from} = k_2\lrt{B_\to-B_\from}
	\end{array}}
\end{equation}
nel caso pi\`u generico.

Nel nostro caso consideriamo un'onda piana da $-\infty$ che si muove verso destra prima di incontrare la barriera:
\begin{equation}
\lr\{.{\begin{array}{ccc}
	A_\to = 1&,&
	A_\from = r\\
	B_\to = t&,&
	B_\from=0
	\end{array}} \Rightarrow\lr\{.{\begin{array}{l}
	1+r = t\\
	k_1 \lrt{1-r} = k_2 t
	\end{array}}
\end{equation}

Risolvendo il sistema otteniamo:
\begin{equation}
	\lr\{.{\begin{array}{l}
		r = \frac{k_1-k_2}{k1+k_2}\\
		t = \frac{2k_1}{k1+k_2}
	\end{array}}
\end{equation}
Quelli che mi interessano sono per\`o il coefficiente di riflessione e trasmissione, dato che stiamo parlando di funzioni con norma L2 sono:
\begin{equation}
\lr\{.{\begin{array}{l}
	R = \lr||{r}^2 = \frac{\lrt{k_1-k_2}^2}{\lrt{k1+k_2}^2}\\
	T = 1-R = \lr||{t}^2\frac{k_2}{k_1} = \frac{4 k_1 k_2}{\lrt{k1+k_2}^2}
	\end{array}}
\end{equation}

Metto i risultati in termini di $E$ e $V$

\begin{equation}\label{eq:riflessioneStepvanilla}
\lr\{.{\begin{array}{l}
	R =  \frac{2E -V -2 \sqrt{E\lrt{E-V}}}{2E -V +2 \sqrt{E\lrt{E-V}}} = \frac{2E/V -1 -2 \sqrt{E/V\lrt{E/V-1}}}{2E/V -1 +2 \sqrt{E/V\lrt{E/V-1}}}\\
	T =  \frac{4 \sqrt{E\lrt{E-V}}}{2E -V +2 \sqrt{E\lrt{E-V}}} = \frac{4 \sqrt{E/V\lrt{E/V-1}}}{2E/V -1 +2 \sqrt{E/V\lrt{E/V-1}}}
	\end{array}}
\end{equation}

\begin{figure}[h]
	\centering
	\begin{tikzpicture}
	\begin{axis}[axis x line = center, axis y line = center, xlabel=E/V, xlabel style={above}, xmin = -0.1, xmax = 3.1, ymin=-0.1,ymax =1.1,legend style={at={(1,0.5)},anchor=east}]
	\addplot[domain=0:3, samples =501, red]		plot[id=Trasm]	function{\Tjump(x)};
	\addlegendentry{Trasmissione}
	\addplot[domain=0:3, samples =501, blue]	plot[id=Refl]	function{\Rjump(x)};
	\addlegendentry{Riflessione}
	\end{axis}
	\end{tikzpicture}
	\caption{Coefficienti di trasmissione e riflessione}
\end{figure}

\subsection{Calcoli Barriera}\label{sec:BarrieraCalc}
Con un potenziale del tipo \eqref{eq:Barriera} ho tre possibilit\`a:
\begin{description}
	\item[$E<V_0$] per cui ho $k_2 = i \sqrt\frac{2m \lrt{V_0-E}}{\hbar^2}$
	\item[$E>V_0$] per cui ho $k_2 = \sqrt\frac{2m \lrt{E-V_0}}{\hbar^2}$
	\item[$E=V_0$] per cui ho $k_2 = 0$
\end{description}

Prima di separare i casi impongo la continuit\`a di funzione e derivata nel caso  $E \neq V_0$:
\begin{equation}\label{eq:continuita}
\lr\{.{\begin{array}{lr}
	A_\to e^{ik_1x} + A_\from e^{-ik_1x}	&	x<0\\
	C_\to e^{ik_2x} + C_\from e^{-ik_2x}	&	0<x<a\\
	B_\to e^{ik_1x} + B_\from e^{-ik_1x}	&	x>a
	\end{array}}\Rightarrow\lr\{.{\begin{array}{lr}
	A_\to  +A_\from = C_\to+C_\from															&	\text{in }0\\
	B_\to e^{ik_1a} +B_\from e^{-ik_1a} = C_\to e^{ik_2a}+C_\from e^{-ik_2a}				&	\text{in }a\\
	k_1\lrt{A_\to-A_\from}=k_2\lrt{C_\to-C_\from }	&	\text{in }0\\
	k_1\lrt{B_\to e^{ik_1a}-B_\from e^{-ik_1a}} = k_2\lrt{C_\to e^{ik_2a}-C_\from e^{-ik_2a}}	&	\text{in }a
	\end{array}}
\end{equation}
nel caso $E \neq V_0$. Consideriamo un'onda piana da $-\infty$ che si muove verso destra prima di incontrare la barriera:
\begin{equation}
\lr\{.{\begin{array}{ccc}
	A_\to = 1&,&
	A_\from = r\\
	B_\to = t&,&
	B_\from=0\\
	C_\to = m_1&,&
	C_\from=m_2
	\end{array}} \Rightarrow\lr\{.{\begin{array}{l}
	1+r= m_1 +m_2 \\
	t e^{ik_1a} = m_1 e^{ik_2a}+m_2 e^{-ik_2a}\\
	k_1\lrt{1 - r}=k_2\lrt{m_1 - m_2}\\
	k_1 t e^{ik_1a} = k_2\lrt{m_1 e^{ik_2a}-m_2 e^{-ik_2a}}
	\end{array}}
\end{equation}

Risolvendo il sistema ottengo:
\begin{equation}
\lr\{.{\begin{array}{rcl}
	r	&=& \frac{\lrt{e^{2 i a k_2}-1}\lrt{k_1^2-k_2^2}}{\lrt{k_1-k_2}^2 e^{2 i a k_2}-\lrt{k_1+k_2}^2}\\
	t	&=& -\frac{4 e^{-i a \lrt{k_1-k_2}} k_1 k_2}{\lrt{k_1-k_2}^2 e^{2 i a k_2}-\lrt{k_1+k_2}^2}\\
	m_1	&=& \frac{2k_1\lrt{k_1+k_2}}{\lrt{k_1+k_2}^2-\lrt{k_1-k_2}^2 e^{2 i a k_2}}\\
	m_2	&=& \frac{2 e^{2 i a k_2}k_1\lrt{k_2-k_1}}{\lrt{k_1+k_2}^2-\lrt{k_1-k_2}^2 e^{2 i a k_2}}
	\end{array}}
\end{equation}

ma a noi interessa il valore del modulo quadro della funzione, e quindi:

\begin{equation}
\lr\{.{\begin{array}{l}
	R = \lr||{r}^2 = \lr||{\frac{\sin\lrt{a k_2}\lrt{k_1^2-k_2^2}}{\lrt{k_1^2+k_2^2} \sin\lrt{a k_2}+2 i k_1 k_2 \cos\lrt{a k_2}}}^2=
	\frac{\lrt{k_1^2-k_2^2}^2\sin^2\lrt{a k_2}}{4 k_1^2 k_2^2 \cos^2\lrt{a k_2}+\lrt{k_1^2+k_2^2}^2\sin^2\lrt{a k_2}}\\
	T = \lr||{t}^2 = \lr||{\frac{-2e^{-i a k_1}k_1 k_2}{\lrt{k_1^2+k_2^2} \sin\lrt{a k_2}+2 i k_1 k_2 \cos\lrt{a k_2}}}^2=
	\frac{4 k_1^2 k_2^2}{4 k_1^2 k_2^2 \cos^2\lrt{a k_2}+\lrt{k_1^2+k_2^2}^2\sin^2\lrt{a k_2}}
	\end{array}}
\end{equation}

A questo punto esplicito in termini di $E$ e $V$ ed $m$, se $E>V$:
%k_1 = \sqrt{2m E}
%k_2 = \sqrt{2m \lrt{E-V_0}}
\begin{equation}
\lr\{.{\begin{array}{l}
	R = \lrt{1+4 \frac{\lrt{E^2 - EV_0}}{V_0^2\sin^2\lrt{a \sqrt{2m \lrt{E-V_0}/\hbar^2}}}}^{-1}\\
	T = \lrt{1 +\frac{V_0^2\sin^2\lrt{a \sqrt{2m \lrt{E-V_0}/\hbar^2}}}{4\lrt{E^2 - EV_0}} }^{-1}
	\end{array}}
\end{equation}

E se $E<V$, dato  che $\sqrt{-x} = i\sqrt{x}$ e sapendo che $\sin\lrt{i x} = i\sinh\lrt{x}$:

\begin{equation}
\lr\{.{\begin{array}{l}
	R = \lrt{1+4 \frac{\lrt{EV_0-E^2}}{V_0^2\sinh^2\lrt{a \sqrt{2m \lrt{V_0-E}/\hbar^2}}}}^{-1}\\
	T = \lrt{1 +\frac{V_0^2\sinh^2\lrt{a \sqrt{2m \lrt{V_0-E}/\hbar^2}}}{4\lrt{EV_0-E^2}} }^{-1}
	\end{array}}
\end{equation}

Prima di proseguire devo risolvere l'equazione nell'ultimo caso, $E=V_0$. In questo caso mi serve sapere che la soluzione di un'equazione del tipo $\pde{^2}{x^2}f=0$ \`e del $f\lrt{x}=C_1 x + C_0$. Riscrivo la \eqref{eq:continuita}:

\begin{equation}
\lr\{.{\begin{array}{lr}
	A_\to e^{ik_1x} + A_\from e^{-ik_1x}	&	x<0\\
	C_1 x + C_0	&	0<x<a\\
	B_\to e^{ik_1x} + B_\from e^{-ik_1x}	&	x>a
	\end{array}}\Rightarrow\lr\{.{\begin{array}{lr}
	A_\to  +A_\from = C_0															&	\text{in }0\\
	B_\to e^{ik_1a} +B_\from e^{-ik_1a} = C_1 a + C_0			&	\text{in }a\\
	i k_1\lrt{A_\to-A_\from} = C_1	&	\text{in }0\\
	i k_1\lrt{B_\to e^{ik_1a}-B_\from e^{-ik_1a}} = C_1	&	\text{in }a
	\end{array}}
\end{equation}

Rifacendo il ragionamento precedente:

\begin{equation}
\lr\{.{\begin{array}{ccc}
	A_\to = 1&,&
	A_\from = r\\
	B_\to = t&,&
	B_\from=0\\
	C_1 = m_1&,&
	C_0 = m_2
	\end{array}} \Rightarrow\lr\{.{\begin{array}{l}
	1+r= m_2 \\
	t e^{ik_1a} = m_1 a + m_2\\
	i k_1\lrt{1 - r} = m_1\\
	i k_1 t e^{ik_1a} = m_1
	\end{array}}
\end{equation}

Risolvendo il sistema ottengo:
\begin{equation}
\lr\{.{\begin{array}{rcl}
	r	&=& \frac{a k_1}{2 i + a k_1}\\
	t	&=& -\frac{2 e^{-i a k_1}}{i a k_1 - 2}\\
	m_1	&=& -\frac{2 k_1}{2 i + a k_1}\\
	m_2	&=& 2+\frac{2}{i a k_1 - 2}
	\end{array}}
\end{equation}

Proseguendo:
\begin{equation}
\lr\{.{\begin{array}{l}
	R = \lr||{r}^2 = \lr||{\frac{a k_1}{2 i + a k_1}}^2=
	\lrt{1+\frac{4}{a^2k_1^2}}^{-1}\\
	T = \lr||{t}^2 = \lr||{\frac{2 e^{-i a k_1}}{i a k_1 - 2}}^2=
	\frac{4}{4+a^2k_1^2}
	\end{array}}
\end{equation}

E in funzione di $E$:
%k_1 = \sqrt{2m E}
%k_2 = \sqrt{2m \lrt{E-V_0}}
\begin{equation}
\lr\{.{\begin{array}{l}
	R = \lrt{1+\frac{2\hbar^2}{ a^2 m E}}^{-1}\\
	T = \lrt{1 + \frac{a^2 m E}{2\hbar^2}}^{-1}
	\end{array}}
\end{equation}

Quindi avr\`o, per $R$:

\begin{equation}\label{eq:riflessionevanilla}
R\lrt{m,E,V_0} = \lr\{.{\begin{array}{lr}
	\lrt{1+4 \frac{\lrt{E^2 - EV_0}}{V_0^2\sin^2\lrt{a \sqrt{2m \lrt{E-V_0}/\hbar^2}}}}^{-1} & E>V_0 \\
	\lrt{1+\frac{2\hbar^2}{ a^2 m E}}^{-1}                                                   & E=V_0 \\
	\lrt{1+4 \frac{\lrt{EV_0-E^2}}{V_0^2\sinh^2\lrt{a \sqrt{2m \lrt{V_0-E}/\hbar^2}}}}^{-1}  & E<V_0
\end{array}}
\end{equation}

mentre per $T$:

\begin{equation}
T\lrt{m,E,V_0} = \lr\{.{\begin{array}{lr}
	\lrt{1 +\frac{V_0^2\sin^2\lrt{a \sqrt{2m \lrt{E-V_0}/\hbar^2}}}{4\lrt{E^2 - EV_0}} }^{-1} & E>V_0 \\
	\lrt{1 + \frac{a^2 m E}{2\hbar^2}}^{-1}                                                   & E=V_0 \\
	\lrt{1 +\frac{V_0^2\sinh^2\lrt{a \sqrt{2m \lrt{V_0-E}/\hbar^2}}}{4\lrt{EV_0-E^2}} }^{-1}  & E<V_0
\end{array}}
\end{equation}
