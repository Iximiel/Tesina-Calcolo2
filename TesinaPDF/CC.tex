\section{Condizioni al contorno}
In seguito espongo come \`e possibile adottare alcune condizioni al contorno:
\begin{itemize}
\item Dirichlet: Conosco i valori della funzione negli estremi del dominio
\item Neumann: Conosco i valori della derivata della funzione negli estremi del dominio
\item Robin: Conosco una combinazione lineare tra il valore della funzione e la sua derivata negli estremi del dominio
%\item Cauchy: Conosco il valore della funzione \textbf{E} il valore della derivata negli estremi del dominio
\item Miste: Negli estremi ho tipi differenti di condizioni al contorno
\end{itemize}
\subsection{Dirichlet}
Conosco il valore della funzione negli estremi del dominio.
\begin{equation}
F(x,t) = f(x,t) \forall x \in \partial D
\end{equation}

Assegno a $F_0^{j+1}$ e $F_{N-1}^{j+1}$ il valore noto, e` quindi inutile calcolare la prima e l'ultima riga della matrice $N\times N$ e posso trattare tutto come se la matrice fosse $N-2\times N-2$, con indici da $1$ a $N-2$ per tenere conto delle condizioni devo cambiare i valori:
\begin{equation}
\begin{array}{ll|ll}
	a'_0 = 0 & ak'_0= 0&	a'_1 = 0 & ak'_1= ak_1\\
	d_0' = 1& dk'_0 = 0&	d_1' = d_1& dk'_1 = dk_1\\
	c_0' =  0& ck_0' = 0&	c_1' =  c_1& ck_1' = ck_1\\
	e_0' = -F_0^{j+1} & ek_0' = 0&	e_1' = e_1+a_1 F^{j+1}_0 & ek_1' = ek_1
\end{array}
\end{equation}
Che equivale  a scrivere:
\begin{equation}
\begin{array}{l l l}
b_0' = F_0^{j+1}&h_0' = 0&p_0' = F_0^{j+1}\\
b_1' = b_1 - a_1 F_0^{j+1}  &h_1' = \frac{c_1}{d_1'}&p_1' = \frac{b_1'}{d_1'}
\end{array}
\end{equation}
Di conseguenza $F_1^{j+1} = p_1' + h_1'F_2^{j+1}$ e $F_0^{j+1} = p_0' + h_0' F_1^{j+1} = F_0^{j+1}$. Mentre se la condizione si presenta per l'ultimo punto del dominio:
\begin{equation}
\begin{array}{ll|ll}
a'{N-2} = a_{N-2} & ak'_{N-2}= ak_{N-2} &	a'_{N-1} = 0 & ak'_{N-1}= 0\\
d_{N-2}' = d_{N-2}& dk'_{N-2} = dk_{N-2}&	d_{N-1}' = 1& dk'_{N-1} = 0\\
c_{N-2}' =  0& ck_{N-2}' = ck'_{N-2}	&	c_{N-1}' =  0& ck_{N-1}' = 0\\
e_{N-2}' = e_{N-2}+c_{N-2}F_{N-1}^{j+1} & ek_{N-2}' = - F^{j+1}_{N-1}&	e_{N-1}' =  -F^{j+1}_{N-1} & ek_{N-1}' = 0
\end{array}
\end{equation}
Che posso riscrivere:
\begin{equation}
\begin{array}{l l l}
b_{N-2}' = b_{N-2} - c_{N-2} F_{N-1}^{j+1}  &h_{N-2}' = 0&p_{N-2}' = p_{N-2}\\
b_{N-1}' = F_{N-1}^{j+1}&h_{N-1}' = 0&p_{N-1}' = F_0^{j+1}
\end{array}
\end{equation}
e di conseguenza $F_{N-1}^{j+1} = p_{N-1}' = F_{N-1}^{j+1}$ e $F_{N-2}^{j+1} = p_{N-2}'+h_{N-2}' F_{N-1}^{j+1} =  p_{N-2}'$.

\subsection{Neumann}
Conosco il valore della derivata negli estremi del dominio.
\begin{equation}
\pde Fx(x,t) = f(x,t) \forall x \in \partial D
\end{equation}
La spiegazione e l'esempio per questa risoluzione lo fornisco nel paragrafo dedicato a Robin.

\subsection{Robin}
Conosco una combinazione lineare tra la derivata e il valore della funzione negli estremi del dominio.
\begin{equation}\label{eq:Robin}
\pde Fx(x,t) +r(x,t) F(x,t) = f(x,t) \forall x \in \partial D
\end{equation}

Come prima, per mantenere la precisione del metodo ($\Delta x^2$) non posso usare la definizione  centrale,  ho quindi bisogno di inventarmi un ''nodo fantasma'' $F_{-1}^{j+1}$ (o $F_{N}^{j+1}$ se fosse la condizione nell'ultimo punto del dominio):
\begin{equation}
\pde Fx(x(i=0),t(j=j+1)) = \frac{F_{1}^{j+1}-F_{-1}^{j+1}}{2\Delta x}
\end{equation}
Prendo la \eqref{eq:Robin} e faccio le adeguate sostituzioni per $i=0$ e al tempo geerico $j=n$:
\begin{equation}
\frac{F_{1}^{n}-F_{-1}^{n}}{2\Delta x} + R_0^nF_{0}^{n} = f_0^n  \to 
F_{-1}^{n} = F_{1}^{n} + 2\Delta x \lrt{R_0^nF_{0}^{n}-f^n_0}
\end{equation}

Partendo dalla forma matriciale del problema generico:
\begin{equation}
a_0  F_{-1}^{j+1} + d_0 F_{0}^{j+1} + c_{0}F_{1}^{j+1} + e_0 = 
ak_0 F_{-1}^{j}   + dk_0 F_{0}^{j}  + ck_0 F_{1}^{j} + ek_0
\end{equation}

e sostituendo i vari $F_{-1}^{n}$:
\begin{equation}
a_0 \lrt{ F_{1}^{j+1} + 2 \Delta x \lrt{R_0^{j+1}F_{0}^{j+1}-f^{j+1}_0}} +d_0 F_{0}^{j+1} +c_{0}F_{1}^{j+1} = 
ak_0 \lrt{F_{1}^{j} + 2 \Delta x \lrt{R_0^jF_{0}^{j}-f^{j}_0}} + dk_0 F_{0}^{j} + ck_0 F_{1}^{j}
\end{equation}

A questo punto raccolgo i termini con lo stesso punto della funzione:
\begin{equation}
\lrt{d_0 + 2 a_0 R_0^{j+1}\Delta x} F_0^{j+1} + \lrt{c_0+a_0}F^{j+1}_1 
- 2 a_0 f^{j+1}_0 \Delta x = 
\lrt{dk_0 + 2 ak_0 R_0^{j}\Delta x} F_0^{j} + \lrt{ck_0+ak_0}F^{j}_1 
- 2 ak_0 f^j_0 \Delta x
\end{equation}
E quindi i parametri interessati della matrice diventano:
\begin{equation}
\begin{array}{ll}
a'_0 = 0 & ak'_0= 0\\
d_0' = d_0 + 2 a_0 R_0^{j+1}\Delta x& dk'_0 = dk_0 + 2 ak_0 R_0^{j}\Delta x\\
c_0' = a_0+c_0 & ck_0' = ak_0+ck_0\\
e_0' = e_0 - 2 a_0  f^{j+1}_0 \Delta x & ek_0' = ek_0 - 2 ak_0  f^{j}_0 \Delta x
\end{array}
\end{equation}
Che equivale a scrivere:
\begin{equation}
\begin{array}{l}
b_0' = dk'_0 F_0^j  + ck_0' F_1^j + 2 \Delta x \lrt{a_0 f^{j+1}_0-ak_0 f^j_0}\\
h_0' = \frac{c_0'}{d_0'}\\
p_0' = \frac{b_0'}{d_0'}
\end{array}
\end{equation}
e di conseguenza $F_0^{j+1} = p_{0}' + h_0' F_1^{j+1}$.

Mentre se la condizione si presenta come ultimo punto del dominio:
\begin{equation}
\frac{F_{N}^{n}-F_{N-2}^{n}}{2\Delta x} + R_{N-1}^nF_{N-1}^{n} = f_{N-1}^n  \to 
F_{N}^{n} = F_{N-2}^{n} - 2\Delta x \lrt{R_{N-1}^nF_{N-1}^{n}-f^n_{N-1}}
\end{equation}

Salto i passaggi, molto simili a quelli della spiegazione precedente,
per il calcolo di $b_{N-1}'$ andranno usati i seguenti parametri:
\begin{equation}
\begin{array}{ll}
a_{N-1}' = a_{N-1}+c_{N-1} & ak_{N-1}' = ak_{N-1}+ck_{N-1}\\
d_{N-1}' = d_{N-1} - 2 c_{N-1} R_{N-1}^{j+1}\Delta x& dk'_{N-1} = dk_{N-1} - 2 ck_{N-1} R_{N-1}^{j}\Delta x\\
c'_{N-1} = 0 & ck'_{N-1}= 0\\
e'_{N-1} = e_{N-1} + c_{N-1} f^{j+1}_{N-1}\Delta x & ek'_{N-1} = ek_{N-1} + ck_{N-1} f^{j+1}_{N-1}\Delta x
\end{array}
\end{equation}
Che equivale a scrivere:
\begin{equation}
\begin{array}{l}
b_{N-1} = ak_{N-1}' F_{N-2}^j + dk'_{N-1} F_{N-1}^j - 2 \Delta x \lrt{c_{N-1} f^{j+1}_{N-1}-ck_{N-1} f^j_{N-1}}\\
h_{N-1}' = 0\\
p_{N-1}' = \frac{b_{N-1}' + a_{N-1}'p_{N-2}}{d_{N-1}' - a_{N-1}'h_{N-2}}
\end{array}
\end{equation}

e di conseguenza $F_{N-1}^{j+1} = p_{N-1}'$, anche perch\`e \`e il primo punto da cui si parte per calcolare il valore della funzione in $j+1$.

Se faccio in modo di eliminare il coefficiente che moltiplica il valore della funzione (gli $R$) ottengo le condizioni a  contorno di Neuman.
\subsection{Osservazione}
Il modo in cui ho trattato i parametri per quanto riguarda le condizioni al contorno di Dirichlet nei punti $0$ e $N-1$, non \`e matematicamente correttissimo infatti i parametri andrebbero messi tutti a 0 in quanto quei punti non fanno parte dell'algoritmo. Ho impostato i valori  per avere un algoritmo che possa svolgere il calcolo rispettando le condizioni al contorno senza sapere quali siano, mettendo nelle mani dell'utente che si occuper\`a di impostare i corretti parametri della matrice la gestione delle condizioni.