\documentclass[]{article}
\usepackage[cm]{fullpage}
\usepackage{amsmath}
\usepackage{amsfonts}
\usepackage{amssymb}

\usepackage{showlabels} % debug, togliere

%\usepackage[italian]{babel}
%\usepackage[latin1]{inputenc}
%\usepackage[utf8x]{inputenc}
\usepackage{dsfont}
%\usepackage{amsthm}
%\usepackage[cm]{fullpage}
\usepackage{enumerate}
%\usepackage{extarrows}
%\usepackage{mathrsfs}
%\usepackage{braket}
%\usepackage{wrapfig}
\usepackage{tikz}
%\usepackage{verbatim}

\usepackage{hyperref}%deve essere l'ultimo package
%http://www.tug.org/applications/hyperref/manual.html
\hypersetup{colorlinks=true,
linkcolor=black}

%derivate
\newcommand{\de}[2]{\ensuremath{\frac{\mathrm{d} #1}{\mathrm{d} #2}}}
\newcommand{\lde}[2]{\ensuremath{{\mathrm{d} #1}/{\mathrm{d} #2}}}
\newcommand{\dt}[2]{\ensuremath{\frac{\mathrm{d} #1}{\mathrm{d} t}}}
\newcommand{\Dt}[1]{\ensuremath{\frac{\mathrm{D} #1}{\mathrm{D} t}}}
\newcommand{\pde}[2]{\ensuremath{\frac{\partial #1}{\partial #2}}}
\newcommand{\lpde}[2]{\ensuremath{{\partial #1}/{\partial #2}}}
%lettere
\newcommand{\df}{\ensuremath{\mathrm{d}}}
\newcommand{\w}{\ensuremath{\omega}}
\newcommand{\R}{\mathds{R}}
\renewcommand{\O}{\mathcal{O}} %ordine di
%parentesi
\newcommand{\lr}[3]{\ensuremath{\left#1 #3 \right#2}}
\newcommand{\lrt}[1]{\lr{(}{)}{#1}}
\newcommand{\lrq}[1]{\lr{[}{]}{#1}}
\newcommand{\lrg}[1]{\lr{\{}{\}}{#1}}
\newcommand{\media}[1]{\lr{<}{>}{#1}}

\newcommand{\infint}{\int\limits_{-\infty}^{+\infty}}
\newcommand{\Schrodinger}{Schr\"{o}dinger }

%vettori
\renewcommand{\vec}[1]{\boldsymbol{#1}}

\numberwithin{equation}{subsection}

%opening
\title{}
\author{Daniele Rapetti}
\date{}

%\makeindex

\begin{document}
\section{Introduzione: matematica}
\subsection{Derivate numeriche}
Per prima cosa inizio con un piccolo elenco di derivate numeriche, usando il metodo delle differenze numeriche:

La derivata prima (in avanti) \`e:
\begin{equation}
\pde{F}x(a) \simeq \frac{F(a+h)-F(a)}{h} + O(h)
\end{equation}
Ma la sua precisione \`e al primo ordine, per cui utilizzer\`o la versione cosiddetta ''centrale'':
\begin{equation}
\pde{F}x(a) \simeq \frac{F(a+h)-F(a-h)}{2*h} + O(h^2)
\end{equation}

Per la derivata seconda il discorso e` simile ma in entrambi i casi la precisione \`e sempre al secondo ordine:
\begin{equation}
\pde{^2F}{x^2}(a) \simeq \frac{\pde{F}x(a+h)-\pde{F}x(a)}{h} = \frac{F(a+2h)+F(a)-2F(a+h)}{h^2} + O(h^2)
\end{equation}

\begin{equation}
\pde{^2F}{x^2}(a) \simeq  \frac{F(a+h)+F(a-h)-2F(a)}{h^2} + O(h^2)
\end{equation}
\subsection{Equazione del calore}
Per prima cosa parto dall'equazione del calore
\begin{equation}\label{eq:calore}
\pde Tt =K \pde{^2T}{x^2}
\end{equation}
Per calcolare l'equazione come prima cosa calcoliamo le derivate:

\begin{equation}
\pde Tt\lrt{x,t} = \frac{T(x,t+\Delta t)-T(x,t)}{\Delta t}
\end{equation}
e
\begin{equation}
\pde{^2T}{x^2}(x,t) = \frac{T(x+\Delta x,t)+T(x-\Delta x,t)-2T(x,t)}{\Delta x^2}
\end{equation}

Per prima cosa eguaglio le derivate (ma devo ricordare che la risoluzione che si avrebbe precisione $\Delta x^2$ nello spazio e $\Delta t$ nel tempo):
\begin{equation}
T(x,t+\Delta t) = k\frac{\Delta t}{\Delta x^2} \lrt{T(x+\Delta x,t)+T(x-\Delta x,t)-2T(x,t)} + T(x,t)
\end{equation}
e andando avanti:
\begin{equation}
\frac{T(x,t+\Delta t)-T(x,t)}{\Delta t} = k \frac{T(x+\Delta x,t)+T(x-\Delta x,t)-2T(x,t)}{\Delta x^2}
\end{equation}
A questo punto calcolo ogni punto conoscendo i tre del passo precedente, ma non aproffondir\`o la cosa.

Oppure posso mettere ugualiare con la derivata spaziale all'istante sucessivo:
\begin{equation}
\frac{T(x,t+\Delta t)-T(x,t)}{\Delta t} = k \frac{T(x+\Delta x,t+\Delta t)+T(x-\Delta x,t+\Delta t)-2T(x,t+\Delta t)}{\Delta x^2}
\end{equation}
proseguo:
\begin{equation}
T(x,t+\Delta t)- k \frac{\Delta t}{\Delta x^2}\lrt{T(x+\Delta x,t+\Delta t)+T(x-\Delta x,t+\Delta t)-2T(x,t+\Delta t)} = T(x,t)
\end{equation}

Anche in questo caso non approfondisco, preferisco ricavare il metodo di Crank-Nicolson per poi generalizzarlo e descriverne l'algoritmo utilizzato per la risoluzione dell'equazione:

\subsection{Un'esempio di come ricavare il metodo di Crank Nicolson: l'equazione del calore}
Innanzitutto utilizzamo la definizione centrale della derivata prima in modo da avere una precisione di $\Delta t^2$ anche per quanto riguarda il tempo, ma la calcolo in $t+\Delta t/2$ con incremento $\Delta t$:
\begin{equation}
\pde{T}t(x,t+\Delta t/2) \simeq \frac{T(x,t+\Delta t/2+\Delta t/2)-T(x,t-\Delta t/2+\Delta t/2)}{2*\Delta t/2} = \frac{T(x,t+\Delta t)-T(x,t)}{\Delta t}
\end{equation}
Per lo spazio utilizziamo le derivate calcolate negli esempi precedenti.
A questo punto abbiamo $\pde{T}t(x,t+\Delta t/2)$, $\pde{^2T}{x^2}(x,t)$ e $\pde{^2T}{x^2}(x,t+\Delta t)$ per rispettare l'equazione facciamo la media delle due derivate spaziali ai tempi $t$ e $t+\Delta t$.
\begin{equation}\label{eq:HeatForCrank}
\begin{aligned}
\frac{T(x,t+\Delta t)-T(x,t)}{\Delta t} = \frac K2 &\lr(.{\frac{T(x+\Delta x,t)+T(x-\Delta x,t)-2T(x,t)}{\Delta x^2}}\\
&\lr.){+\frac{T(x+\Delta x,t+\Delta t)+T(x-\Delta x,t+\Delta t)-2T(x,t+\Delta t)}{\Delta x^2}}
\end{aligned}
\end{equation}

In pochi passaggi si arriva a separare le parti a tempo differente:, con $\eta = K\frac{\Delta t}{\Delta x^2}$:
\begin{equation}
\lrt{\frac 2\eta +2}T(x,t+\Delta t) -T(x+\Delta x,t+\Delta t)-T(x-\Delta x,t+\Delta t) = \lrt{\frac 2\eta -2}T(x,t)+T(x+\Delta x,t)+T(x-\Delta x,t)
\end{equation}
Per ogni istante di tempo ho una matrice tridiagonale per il passo temporale che conosco e per quello successivo. In \autoref{section:soluzionetri} descriver\`o come si risolve una di queste matrici.

\section{Costruzione dell'algoritmo}
\subsection{Ottenere il sistema}
Ho fatto un esempio col l'equazione del calore. Prima di procedere alla spiegazione su come si semplifica e si risolve un sistema di equazioni riconducibile ad una matrice tridiagonale spiegher\`o come condurre la PDE pi\`u generale ad un sistema del genere.

Partendo da una generica equazione di secondo ordine:
\begin{equation}\label{eq:generica}
\partial_t F = D_2 \partial^2_x F + D_1 \partial_x F + V(x,t) F + U(x,t)
\end{equation}

Il coefficiente della derivata temporale \`e ignorato perch\'e \`e incluso negli altri coefficienti, ovviamente $D_2$, il coefficiente della derivata seconda spaziale non deve essere mai uguale a zero!
Inoltre preferisco lasciare $D_1$ e $D_2$ come costanti nel tempo e nello spazio, e lascio la dipendenza temporale e spaziale a $V$ e $U$, che possono essere costanti a loro volta.

In seguito indicher\`o la il passo nello spazio a pedice con $i$ e quello nel tempo in apice con $j$.
Discretizzando l'equazione ottengo quindi (ho mantenuto la definizione di derivata centrale anche per quella prima spaziale in modo da mantenere la precisione):
\begin{equation}
\begin{aligned}
\frac{F_i^{j+1} - F_i^j}{\Delta t} = \frac 12&\lr(.{D_2\frac{F^j_{i+1}+F^{j}_{i-1}-2F_i^{j}}{\Delta x^2} + D_1\frac{F^j_{i+1}-F^{j}_{i-1}}{2\Delta x} + F_i^j V_i^j + U_i^j+}\\
&\lr.){D_2\frac{F^{j+1}_{i+1}+F^{j+1}_{i-1}-2F_i^{j+1}}{\Delta x^2} + D_1\frac{F^{j+1}_{i+1}-F^{j+1}_{i-1}}{2\Delta x} + F_i^{j+1} V_i^{j+1} + U_i^{j+1}}
\end{aligned}
\end{equation}
Salto i passaggi e indico con $\eta = \frac {D_2 \Delta t}{\Delta x^2}$ posso scrivere:
\begin{equation}
\begin{aligned}
\lrt{-\eta+D_1\frac{\Delta t}{2\Delta x}}F_{i-1}^{j+1} + \lrt{2 + 2\eta- \Delta t V_i^{j+1}} F^{j+1}_i + \lrt{-\eta-D_1\frac{\Delta t}{2\Delta x}}F_{i+1}^{j+1} - \Delta t U_i^{j+1} = \\
\lrt{\eta-D_1\frac{\Delta t}{2\Delta x}}F_{i-1}^{j} + \lrt{2-2\eta+\Delta t V_i^{j}} F^{j}_i + \lrt{\eta +D_1\frac{\Delta t}{2\Delta x}}F_{i+1}^{j} + \Delta t U_i^{j}
\end{aligned}
\end{equation}

Per rendere pi\`u chiara spiegazione e risoluzione della matrice tridiagonale riassumo l'equazione precedente in:

\begin{equation}\label{eq:pararametri}
\begin{array}{ll r}
a_i^j = -1+\frac{D_1}{D_2}\frac{\Delta x}2             & ak_i^j =1-\frac{D_1}{D_2}\frac{\Delta x}2&\text{parametri dei }(F^*_{i-1})\\
d_i^j = \frac1\eta\lrt{2-\Delta t V_i^{j+1}} +2 & dk_i^j = \frac1\eta\lrt{2+\Delta tV_i^{j}} -2 &\text{parametri dei }(F^*_{i})\\
c_i^j = -1-\frac{D_1}{D_2}\frac{\Delta x}2             & ck_i^j =1+\frac{D_1}{D_2}\frac{\Delta x}2&\text{parametri dei }(F^*_{i+1})\\
e_i^j = -\frac{\Delta x^2}{D_2} U_i^{j+1}              & ek_i^j =\frac{\Delta x^2}{D_2} U_i^{j}&\text{funzioni esterne}
\end{array}
\end{equation}
\subsection{La matrice Tridiagonale: soluzione}\label{section:soluzionetri}
Per ogni istante di tempo $j$ ho un sistema di $N$ equazioni nella forma:
\begin{equation}
a_{i} F_{i-1}^{j+1}+d_i F_{i}^{j+1} +c_{i}F_{i+1}^{j+1}  + e_i= 
ak_i F_{i-1}^{j}+ dk_i F_{i}^{j} + ck_i F_{i+1}^{j} + ek_i
\end{equation}

Dove $j$ rappresenta l'istante di tempo che conosco e $j+1$ quello che sto calcolando. La prima cosa da fare \`e portare nel membro a destra tutte le cose che conosco, dando per scontato che l'unica incognita dell'equazione \`e la funzione:
\begin{equation}
a_{i} F_{i-1}^{j+1}+d_i F_{i}^{j+1} +c_{i}F_{i+1}^{j+1}= 
ak_i F_{i-1}^{j}+ dk_i F_{i}^{j} + ck_i F_{i+1}^{j} + ek_i - e_i
\end{equation}

Per proseguire con la risoluzione \`e meglio passare alla rappresentazione matriciale del sistema (rappresento gli N punti rispettando le convenzioni del C, quindi $i=0\to N-1$):
\begin{equation}
\lrt{\begin{array}{cccccc}
d_0&c_0&&&\\
a_1&d_1&c_1&\\
&&...&&&\\
&&&a_{N-2}&d_{N-2}&c_{N-2}\\
&&&&a_{N-1}&d_{N-1}\\
\end{array}} F^{j+1} = 
\lrt{\begin{array}{cccccc}
dk_0&ck_0&&&&\\
ak_1&dk_1&ck_1&&&\\
&&...&&&\\
&&&ak_{N-2}&dk_{N-2}&ck_{N-2}\\
&&&&ak_{N-1}&dk_{N-1}\\
\end{array}} F^{j} + 
\lrt{\begin{array}{c}
ek_0 - e_0\\
ek_1 - e_1\\
...\\
ek_{N-2} - e_{N-2}\\
ek_{N-1} - e_{N-1}\\
\end{array}}
\end{equation}

Per comodit\`a compatto il lato conosciuto in un vettore $B^j$ le cui componenti sono:
\begin{equation}\label{eq:bi}
b_i^j = ak_i F_{i-1}^{j}+ dk_i F_{i}^{j} + ck_i F_{i+1}^{j} + ek_i-e_i
\end{equation}

\begin{equation}
\lrt{\begin{array}{cccccc}
d_0&c_0&&&\\
a_1&d_1&c_1&\\
&&...&&&\\
&&&...&&\\
&&&a_{N-2}&d_{N-2}&c_{N-2}\\
&&&&a_{N-1}&d_{N-1}\\
\end{array}} F^{j+1} = 
B^j
\end{equation}
A questo punto procedo con il trasformare la matrice nella somma di una matrice identit\`a e di una matrice con valori non nulli solo nelle celle sopra alla diagonale, faccio vedere i primi passaggi:
\begin{equation}
\lrt{\begin{array}{cccccc}
d_0&c_0&0&...&0\\
a_1&d_1&c_1&...&0\\
&.&.&.&&\\
\end{array}} F^{j+1} = \lrt{\begin{array}{c}
b_0\\b_1\\...
\end{array}}\to
\lrt{\begin{array}{cccccc}
1&\frac{c_0}{d_0}&0&...&0\\
a_1&d_1&c_1&...&0\\
&.&.&.&&\\
\end{array}} F^{j+1} = \lrt{\begin{array}{c}
\frac{b_0}{d_0}\\b_1\\...
\end{array}}
\end{equation}
Proseguendo, chiamando $h_0 = \frac{c_0}{d_0}$ e $p_0 = \frac{b_0}{d_0}$, moltiplico la prima riga per $a_1$ e la sottraggo alla seconda, in modo da eliminare $a_1$ dalla seconda riga:
\begin{equation}
\lrt{\begin{array}{cccccc}
1&h_0&0&...&0\\
a_1-a_1 &d_1 -a_1 h_0&c_1&...&0\\
&.&.&.&&\\
\end{array}} F^{j+1} = \lrt{\begin{array}{c}
p_0\\b_1-a_1p_0\\...
\end{array}}\to
\lrt{\begin{array}{cccccc}
1&h_0&0&...&0\\
0 &1&\frac{c_1}{d_1 -a_1 h_0}&...&0\\
&.&.&.&&\\
\end{array}} F^{j+1} = \lrt{\begin{array}{c}
p_0\\\frac{b_1-a_1p_0}{d_1 -a_1 h_0}\\...
\end{array}}
\end{equation}
A questo punto chiamo $h_1 = \frac{c_1}{d_1 -a_1 h_0}$ e $p_1=\frac{b_1-a_1p_0}{d_1 -a_1 h_0}$ e ripeto il ragionamento precedente sottraendo la seconda riga alla terza:
\begin{equation}
\lrt{\begin{array}{cccccc}
1&h_0&0&...&0\\
0 &1&h_1&...&0\\
0&a_3-a_3&d_3-a_3 h_1&c_3&...\\
&.&.&.&&\\
\end{array}} F^{j+1} = \lrt{\begin{array}{c}
p_0\\p_1\\b_3 -a_3 p_1\\...
\end{array}} \to
\lrt{\begin{array}{cccccc}
1&h_0&0&...&0\\
0 &1&h_1&...&0\\
0&0&1&\frac{c_3}{d_3-a_3 h_1}&...\\
&.&.&.&&\\
\end{array}} F^{j+1} = \lrt{\begin{array}{c}
p_0\\p_1\\\frac{b_3 -a_3 p_1}{d_3-a_3 h_1}\\...
\end{array}}
\end{equation}
A questo punto chiamo $h_3 = \frac{c_3}{d_3 -a_3 h_1}$ e $p_3=\frac{b_3-a_3p_1}{d_3 -a_3 h_1}$ e proseguo, ottengo cos\`i le regole:
\begin{equation}\label{eq:hi}
h_i = \frac{c_i}{d_i -a_i h_{i-1}}
\end{equation}
e
\begin{equation}\label{eq:pi}
p_i=\frac{b_i-a_ip_{i-1}}{d_i -a_i h_{i-1}}
\end{equation}

A questo punto ho semplificato il sistema:
\begin{equation}\end{equation}
$$\lrt{\begin{array}{cccccc}
1&h_0&&&\\
&1&h_1&\\
&&...&&&\\
&&&...&&\\
&&&&1&h_{N-2}\\
&&&&&1
\end{array}}F^{j+1} = P$$
\begin{equation}\end{equation}

Per risolvere il sistema devo calcolare il vettore delle $P$ per poi ottenere i valori di $F^{j+1}$ a partire dall'ultimo $F_{N-1}^{j+1} = p_{N-1}$ con la formula:
\begin{equation}
F_{i}^{j+1} = p_{i}+h_i F_{i+1}^{j+1}
\end{equation}
A questo punto ho bisogno di conoscere come trattare le condizioni al contorno.
\section{Condizioni al contorno}
In seguito espongo come \`e possibile adottare alcune condizioni al contorno:
\begin{itemize}
\item Dirichlet: Conosco i valori della funzione negli estremi del dominio
\item Neumann: Conosco i valori della derivata della funzione negli estremi del dominio
\item Robin: Conosco una combinazione lineare tra il valore della funzione e la sua derivata negli estremi del dominio
%\item Cauchy: Conosco il valore della funzione \textbf{E} il valore della derivata negli estremi del dominio
\item Miste: Negli estremi ho tipi differenti di condizioni al contorno
\end{itemize}
\subsection{Dirichlet}
Conosco il valore della funzione negli estremi del dominio.
\begin{equation}
F(x,t) = f(x,t) \forall x \in \partial D
\end{equation}

Assegno a $F_0^{j+1}$ e $F_{N-1}^{j+1}$ il valore noto, e` quindi inutile calcolare la prima e l'ultima riga della matrice $N\times N$ e posso trattare tutto come se la matrice fosse $N-2\times N-2$, con indici da $1$ a $N-2$ per tenere conto delle condizioni devo cambiare i valori:
\begin{equation}
\begin{array}{ll|ll}
	a'_0 = 0 & ak'_0= 0&	a'_1 = 0 & ak'_1= ak_1\\
	d_0' = 1& dk'_0 = 0&	d_1' = d_1& dk'_1 = dk_1\\
	c_0' =  0& ck_0' = 0&	c_1' =  c_1& ck_1' = ck_1\\
	e_0' = -F_0^{j+1} & ek_0' = 0&	e_1' = e_1+a_1 F^{j+1}_0 & ek_1' = ek_1
\end{array}
\end{equation}
Che equivale  a scrivere:
\begin{equation}
\begin{array}{l l l}
b_0' = F_0^{j+1}&h_0' = 0&p_0' = F_0^{j+1}\\
b_1' = b_1 - a_1 F_0^{j+1}  &h_1' = \frac{c_1}{d_1'}&p_1' = \frac{b_1'}{d_1'}
\end{array}
\end{equation}
Di conseguenza $F_1^{j+1} = p_1' + h_1'F_2^{j+1}$ e $F_0^{j+1} = p_0' + h_0' F_1^{j+1} = F_0^{j+1}$. Mentre se la condizione si presenta per l'ultimo punto del dominio:
\begin{equation}
\begin{array}{ll|ll}
a'{N-2} = a_{N-2} & ak'_{N-2}= ak_{N-2} &	a'_{N-1} = 0 & ak'_{N-1}= 0\\
d_{N-2}' = d_{N-2}& dk'_{N-2} = dk_{N-2}&	d_{N-1}' = 1& dk'_{N-1} = 0\\
c_{N-2}' =  0& ck_{N-2}' = ck'_{N-2}	&	c_{N-1}' =  0& ck_{N-1}' = 0\\
e_{N-2}' = e_{N-2}+c_{N-2}F_{N-1}^{j+1} & ek_{N-2}' = - F^{j+1}_{N-1}&	e_{N-1}' =  -F^{j+1}_{N-1} & ek_{N-1}' = 0
\end{array}
\end{equation}
Che posso riscrivere:
\begin{equation}
\begin{array}{l l l}
b_{N-2}' = b_{N-2} - c_{N-2} F_{N-1}^{j+1}  &h_{N-2}' = 0&p_{N-2}' = p_{N-2}\\
b_{N-1}' = F_{N-1}^{j+1}&h_{N-1}' = 0&p_{N-1}' = F_0^{j+1}
\end{array}
\end{equation}
e di conseguenza $F_{N-1}^{j+1} = p_{N-1}' = F_{N-1}^{j+1}$ e $F_{N-2}^{j+1} = p_{N-2}'+h_{N-2}' F_{N-1}^{j+1} =  p_{N-2}'$.

\subsection{Neumann}
Conosco il valore della derivata negli estremi del dominio.
\begin{equation}
\pde Fx(x,t) = f(x,t) \forall x \in \partial D
\end{equation}
La spiegazione e l'esempio per questa risoluzione lo fornisco nel paragrafo dedicato a Robin.

\subsection{Robin}
Conosco una combinazione lineare tra la derivata e il valore della funzione negli estremi del dominio.
\begin{equation}\label{eq:Robin}
\pde Fx(x,t) +r(x,t) F(x,t) = f(x,t) \forall x \in \partial D
\end{equation}

Come prima, per mantenere la precisione del metodo ($\Delta x^2$) non posso usare la definizione  centrale,  ho quindi bisogno di inventarmi un ''nodo fantasma'' $F_{-1}^{j+1}$ (o $F_{N}^{j+1}$ se fosse la condizione nell'ultimo punto del dominio):
\begin{equation}
\pde Fx(x(i=0),t(j=j+1)) = \frac{F_{1}^{j+1}-F_{-1}^{j+1}}{2\Delta x}
\end{equation}
Prendo la \eqref{eq:Robin} e faccio le adeguate sostituzioni per $i=0$ e al tempo geerico $j=n$:
\begin{equation}
\frac{F_{1}^{n}-F_{-1}^{n}}{2\Delta x} + R_0^nF_{0}^{n} = f_0^n  \to 
F_{-1}^{n} = F_{1}^{n} + 2\Delta x \lrt{R_0^nF_{0}^{n}-f^n_0}
\end{equation}

Partendo dalla forma matriciale del problema generico:
\begin{equation}
a_0  F_{-1}^{j+1} + d_0 F_{0}^{j+1} + c_{0}F_{1}^{j+1} + e_0 = 
ak_0 F_{-1}^{j}   + dk_0 F_{0}^{j}  + ck_0 F_{1}^{j} + ek_0
\end{equation}

e sostituendo i vari $F_{-1}^{n}$:
\begin{equation}
a_0 \lrt{ F_{1}^{j+1} + 2 \Delta x \lrt{R_0^{j+1}F_{0}^{j+1}-f^{j+1}_0}} +d_0 F_{0}^{j+1} +c_{0}F_{1}^{j+1} = 
ak_0 \lrt{F_{1}^{j} + 2 \Delta x \lrt{R_0^jF_{0}^{j}-f^{j}_0}} + dk_0 F_{0}^{j} + ck_0 F_{1}^{j}
\end{equation}

A questo punto raccolgo i termini con lo stesso punto della funzione:
\begin{equation}
\lrt{d_0 + 2 a_0 R_0^{j+1}\Delta x} F_0^{j+1} + \lrt{c_0+a_0}F^{j+1}_1 
- 2 a_0 f^{j+1}_0 \Delta x = 
\lrt{dk_0 + 2 ak_0 R_0^{j}\Delta x} F_0^{j} + \lrt{ck_0+ak_0}F^{j}_1 
- 2 ak_0 f^j_0 \Delta x
\end{equation}
E quindi i parametri interessati della matrice diventano:
\begin{equation}
\begin{array}{ll}
a'_0 = 0 & ak'_0= 0\\
d_0' = d_0 + 2 a_0 R_0^{j+1}\Delta x& dk'_0 = dk_0 + 2 ak_0 R_0^{j}\Delta x\\
c_0' = a_0+c_0 & ck_0' = ak_0+ck_0\\
e_0' = e_0 - 2 a_0  f^{j+1}_0 \Delta x & ek_0' = ek_0 - 2 ak_0  f^{j}_0 \Delta x
\end{array}
\end{equation}
Che equivale a scrivere:
\begin{equation}
\begin{array}{l}
b_0' = dk'_0 F_0^j  + ck_0' F_1^j + 2 \Delta x \lrt{a_0 f^{j+1}_0-ak_0 f^j_0}\\
h_0' = \frac{c_0'}{d_0'}\\
p_0' = \frac{b_0'}{d_0'}
\end{array}
\end{equation}
e di conseguenza $F_0^{j+1} = p_{0}' + h_0' F_1^{j+1}$.

Mentre se la condizione si presenta come ultimo punto del dominio:
\begin{equation}
\frac{F_{N}^{n}-F_{N-2}^{n}}{2\Delta x} + R_{N-1}^nF_{N-1}^{n} = f_{N-1}^n  \to 
F_{N}^{n} = F_{N-2}^{n} - 2\Delta x \lrt{R_{N-1}^nF_{N-1}^{n}-f^n_{N-1}}
\end{equation}

Salto i passaggi, molto simili a quelli della spiegazione precedente,
per il calcolo di $b_{N-1}'$ andranno usati i seguenti parametri:
\begin{equation}
\begin{array}{ll}
a_{N-1}' = a_{N-1}+c_{N-1} & ak_{N-1}' = ak_{N-1}+ck_{N-1}\\
d_{N-1}' = d_{N-1} - 2 c_{N-1} R_{N-1}^{j+1}\Delta x& dk'_{N-1} = dk_{N-1} - 2 ck_{N-1} R_{N-1}^{j}\Delta x\\
c'_{N-1} = 0 & ck'_{N-1}= 0\\
e'_{N-1} = e_{N-1} + c_{N-1} f^{j+1}_{N-1}\Delta x & ek'_{N-1} = ek_{N-1} + ck_{N-1} f^{j+1}_{N-1}\Delta x
\end{array}
\end{equation}
Che equivale a scrivere:
\begin{equation}
\begin{array}{l}
b_{N-1} = ak_{N-1}' F_{N-2}^j + dk'_{N-1} F_{N-1}^j - 2 \Delta x \lrt{c_{N-1} f^{j+1}_{N-1}-ck_{N-1} f^j_{N-1}}\\
h_{N-1}' = 0\\
p_{N-1}' = \frac{b_{N-1}' + a_{N-1}'p_{N-2}}{d_{N-1}' - a_{N-1}'h_{N-2}}
\end{array}
\end{equation}

e di conseguenza $F_{N-1}^{j+1} = p_{N-1}'$, anche perch\`e \`e il primo punto da cui si parte per calcolare il valore della funzione in $j+1$.

Se faccio in modo di eliminare il coefficiente che moltiplica il valore della funzione (gli $R$) ottengo le condizioni a  contorno di Neuman.
\subsection{Osservazione}
Il modo in cui ho trattato i parametri per quanto riguarda le condizioni al contorno di Dirichlet nei punti $0$ e $N-1$, non \`e matematicamente correttissimo infatti i parametri andrebbero messi tutti a 0 in quanto quei punti non fanno parte dell'algoritmo. Ho impostato i valori  per avere un algoritmo che possa svolgere il calcolo rispettando le condizioni al contorno senza sapere quali siano, mettendo nelle mani dell'utente che si occuper\`a di impostare i corretti parametri della matrice la gestione delle condizioni.
\section{Applicazione}
\subsection{Equazione del calore}
Riprendiamo l'equazione del calore:
\begin{equation}
\pde T t(x,t) =k\pde{^2}{x^2}T(x,t)
\end{equation}

Per rispettare la convenzione di prima $D_2 = k$ , $D_1 = U = V(x,t) = 0$. 


A questo punto posso sostituire, con $\eta = k\frac{\Delta t}{\Delta x^2}$:
\begin{equation}\label{eq:pararametriHeat}
\begin{array}{ll}
a_i^j = -1            & ak_i^j =1\\
d_i^j = \frac2\eta +2 & dk_i^j = \frac2\eta -2 \\
c_i^j = -1             & ck_i^j =1\\
e_i^j = 0             & ek_i^j =0
\end{array}
\end{equation}
e procedere con i calcoli.
\subsection{Equazione di \Schrodinger}
Lavoriamo con l'equazione di \Schrodinger  dipendente dal tempo:
\begin{equation}
i\hbar\pde \psi t(x,t) =\lrq{-\frac{\hbar^2}{2m}\pde{^2}{x^2}+V(x,t)} \psi(x,t)
\end{equation}
prima di tutto portiamola in una forma tale che non ci sia nulla a moltiplicare la derivata temporale:
\begin{equation}
\pde \psi t(x,t) =\lrq{i\frac{\hbar}{2m}\pde{^2}{x^2}+\frac{\varLambda(x,t)}{i\hbar}} {\psi(x,t)}
\end{equation} 

Per rispettare la convenzione di prima $D_2 = i\frac{\hbar}{2m}$ , $D_1 = U = 0$ e $V(x,t) = \frac{\varLambda(x,t)}{i\hbar}$. Ho usato $\varLambda$ per indicare il potenziale in modo da evitare confusione.


A questo punto posso sostituire, con $\eta = i\frac{\hbar}{2m}\frac{\Delta t}{\Delta x^2}$:
\begin{equation}\label{eq:pararametriSC}
\begin{array}{ll}
a_i^j = -1            & ak_i^j =1\\
d_i^j = \frac1\eta\lrt{2-\Delta t V_i^{j+1}} +2 & dk_i^j = \frac1\eta\lrt{2+\Delta tV_i^{j}} -2 \\
c_i^j = -1             & ck_i^j =1\\
e_i^j = 0             & ek_i^j =0
\end{array}
\end{equation}
\end{document}